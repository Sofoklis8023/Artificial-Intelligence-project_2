\documentclass{article}
\usepackage{graphicx} % Required for inserting images

\title{texnhth nohmosynh 2}
\author{sdi2100107}
\date{December 2023}
\usepackage[english, greek]{babel}
\usepackage[linguistics]{forest}

\begin{document}

\selectlanguage{english} % Switch to English
\maketitle 
\begin{enumerate}
    \item \textbf{Problem 1:} 
    
    \textbf{Solution:}\selectlanguage{greek}\\
    1.1 Ο πρώτος παίχτης έχει 9 δυνατές επιλογές,ο δεύτερος 8,ο τρίτος 7 κ.ο.κ .Επομένως τα παιχνίδια είναι εώς 9! \\
    1.2/1.3\begin{forest}\\
    [\[κενός πίνακας[χ στο0.0[ο στο 0.1][ο στο 0.2][ο στο 1.0][ο στο 1.1][ο στο 1.2][ο στο 2.0][ο στο 2.1][ο στο 2.2]][χ στο 0.1][χ στο 0.2][χ στο 1.0][χ στο 1.1][χ στο 1.2][χ στο 2.0][χ στο 2.1][χ στο 2.2]]]
    \end{forest}\\
    1.4 Η απόφαση θα είναι αυτή που παράγει την μέγιστη $minimax$ τιμή η οποία είναι αυτή που ο παίχτης βάζει την τιμή χ στο κέντρο του πίνακα έτσι ώστε να το Χ1 να ισούται με 4(1 γραμμή,1 στήλη,2 διαγωνίους)\\
    1.5 θα κόψει τους κόμβους δεξία μετά τον κόμβο με την τιμή χ στο κέντρο αν αντιστραφούν οι τιμές θα γίνει ακριβώς το ίδιο
 







    \selectlanguage{english}\item \textbf{Problem 2:} 

    \textbf{Solution:}\\
    \selectlanguage{greek}
    Έστω ότι εξετάζουμε το δέντρο από τα αριστερά προς τα δεξία.Για να έχουμε ελάχιστα κλαδέματα θα πρέπειοι τιμές να είναι σε αύξουσα σειρά απο τα αριστερά προς τα δεξιά,ενώ  για να έχουμε μέγιστα  κλαδέματα σε φθίνουσα σειρά.Επίσης αυτό ισχύει για την περίπτωση που ο κόμβος πρου είναι γονιός στα φύλλα του δέντρου έχει τιμή $min$
    
    
    \selectlanguage{english}\item \textbf{Problem 3:} \\
    \textbf{Solution:} \selectlanguage{greek} \\
    3.α \\1ο επίπεδο κόμβος $max: 3$ \\
    2ο επίπεδο κόμβοι $min: 3 , -2$ \\
    3ο επίπεδο κόμβοι $ max: 3,5,-2,9$\\
    4ο επίπεδο κόμβοι $ min: 1,3,5,-1,-2,-4,7,9$\\
    3.β\\
    Η $minimax$ απόφαση στην ρίζα του δέντρου είναι η ενέργεια προς τα αριστερά στον κόμβο $min$ με την τιμή 3 επειδή οδηγεί στην κατάσταση με την υψηλότερη $Minimax$ τιμή \\
    3.γ\\
    Οι κόμβοι που κλαδεύονται είναι το 8ο φύλλο από τα αριστερά προς τα δεξιά με την τιμή 3 διότι το $a = 5$ και το $b=-1$ και το 12ο φύλλο με την τιμή $-4$ διότι $a=-1$ και $b = -4$
   

    
    
    \selectlanguage{english}\item \textbf{Problem 4:}\\ 
    \textbf{Solution:} \selectlanguage{greek} \\
    4.α Αδύνατο καθώς εφόσον ψάχνουμε για την μέγιστη τιμή κάθε φόρα μπόρει ο επόμενος κόμβος να την έχει,επομένως κλαδευοντάς τον υπάρχει πιθανότητα ο αλγόριθμος να τερματίσει με λανθασμένη τιμή\\ 
    4.β Αδύνατο καθώς αν στο προτελευταίο επιπέδο έχουμε $max$ κόμβους θα συμβεί το 4.α ενώ αν έχουμε $change$ θα πρέπει να ελεγξουμέ όλα τα φύλλα για να υπολογιστεί η τιμή\\
    4.γ Στην περίπτωση που υπάρχει το μήδεν σε κάποιο φύλλο μπορούμε να κλαδέψουμε όλα τα φύλλα μετά από αυτό καθώς είναι αδύνατο να πάρουν τιμή μεγαλύτερη του μηδενός\\
    4.δ Αν ο γονιός των φύλλων είναι $chance$ είναι αδύνατο,αν όμως είναι $max$ και το φύλλο στα αριστερά του είναι μηδέν τότε μπορούμε να κλαδέψουμε τα επόμενα φύλλα του ίδιου γονέα καθώς είναι αδύνατο να πάρουν τιμή μεγαλύτερη του μηδενός\\
    4.ε Αδύνατο κάθως τώρα δεν υπάρχει μέγιστη τιμή όπως ήταν προηγουμένως το μηδέν οπότε πρέπει να εξεταστούν 'ολοι οι κόμβοι\\
    4.στ Αδύνατο για τον ίδιο λόγο με το 4.ε\\
    4.ζ Στην περίπτωση που βρεθεί φύλλο με την τιμή 1 κλαδεύουμε όλα τα επόμενα καθώς δεν είναι δυνατό να έχουν τιμή μεγαλύτερη του 1\\
    4.η Αν ο γονιός των φύλλων είναι $chance$ είναι αδύνατο,αν όμως είναι $max$ και το φύλλο στα αριστερά του είναι 1 τότε μπορούμε να κλαδέψουμε τα επόμενα φύλλα του ίδιου γονέα καθώς είναι αδύνατο να πάρουν τιμή μεγαλύτερη 1\\
    
    
\end{enumerate}

\end{document}